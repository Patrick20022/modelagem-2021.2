\documentclass[12pt,a4paper]{article}
\usepackage[utf8]{inputenc}
\usepackage[brazil]{babel}
\usepackage{amsmath}
\usepackage{listings}
\usepackage[utf8]{inputenc}
\usepackage{makeidx}
\usepackage{listings}
\usepackage{xcolor}
\usepackage{graphicx,url}
\usepackage{subfigure}
\usepackage{float}
\definecolor{codegreen}{rgb}{0,0.6,0}
\definecolor{codegray}{rgb}{0.5,0.5,0.5}
\definecolor{codepurple}{rgb}{0.58,0,0.82}
\definecolor{backcolour}{rgb}{0.95,0.95,0.92}

\lstdefinestyle{mystyle}{
    backgroundcolor=\color{backcolour},   
    commentstyle=\color{codegreen},
    keywordstyle=\color{magenta},
    numberstyle=\tiny\color{codegray},
    stringstyle=\color{codepurple},
    basicstyle=\ttfamily\footnotesize,
    breakatwhitespace=false,         
    breaklines=true,                 
    captionpos=b,                    
    keepspaces=true,                 
    numbers=left,                    
    numbersep=5pt,                  
    showspaces=false,                
    showstringspaces=false,
    showtabs=false,                  
    tabsize=2
}

\lstset{style=mystyle,
literate      =        % Support additional characters
      {á}{{\'a}}1  {é}{{\'e}}1  {í}{{\'i}}1 {ó}{{\'o}}1  {ú}{{\'u}}1
      {Á}{{\'A}}1  {É}{{\'E}}1  {Í}{{\'I}}1 {Ó}{{\'O}}1  {Ú}{{\'U}}1
      {à}{{\`a}}1  {è}{{\`e}}1  {ì}{{\`i}}1 {ò}{{\`o}}1  {ù}{{\`u}}1
      {À}{{\`A}}1  {È}{{\'E}}1  {Ì}{{\`I}}1 {Ò}{{\`O}}1  {Ù}{{\`U}}1
      {ä}{{\"a}}1  {ë}{{\"e}}1  {ï}{{\"i}}1 {ö}{{\"o}}1  {ü}{{\"u}}1
      {Ä}{{\"A}}1  {Ë}{{\"E}}1  {Ï}{{\"I}}1 {Ö}{{\"O}}1  {Ü}{{\"U}}1
      {â}{{\^a}}1  {ê}{{\^e}}1  {î}{{\^i}}1 {ô}{{\^o}}1  {û}{{\^u}}1
      {Â}{{\^A}}1  {Ê}{{\^E}}1  {Î}{{\^I}}1 {Ô}{{\^O}}1  {Û}{{\^U}}1
      {œ}{{\oe}}1  {Œ}{{\OE}}1  {æ}{{\ae}}1 {Æ}{{\AE}}1  {ß}{{\ss}}1
      {ç}{{\c c}}1 {Ç}{{\c C}}1 {ø}{{\o}}1  {å}{{\r a}}1 {Å}{{\r A}}1
      {ã}{{\~a}}1  {õ}{{\~o}}1  {Ã}{{\~A}}1 {Õ}{{\~O}}1
      {ñ}{{\~n}}1  {Ñ}{{\~N}}1  {¿}{{?`}}1  {¡}{{!`}}1
}
\input{settings/Configuracoes_do_Preambulo}

\begin{document}
 

%\maketitle
\thispagestyle{primeira}



\section*{Resumo}
Este artigo propõe um modelo SIR para o Ebola Virus Disease(EVD) usando derivados conformáveis. Discrevemos todas as maneiras possíveis de trasmissão da doença, direta e indireta. 
\section*{Introdução}
O Ebola Virus Disease (EVD) , anteriormente conhecido como febre hemorrágica Ébola, é uma doença grave e fatal causada pela infecção com uma das espécies do vírus do Ébola. O Ébola pode provocar doenças humanos e primatas não-humanos (macacos, gorilas e chimpanzés).
O Ébola é provocado por um vírus da família Filoviridae, género
Ebolavirus. Já foram descobertas cinco espécies do vírus do Ébola.
Quatro delas provocam doenças nos humanos: Vírus do Ébola (Zaire
ebolavirus); vírus do Sudão (Sudan ebolavirus); vírus da floresta
Tai (Tai Forest ebolavirus, anteriormente conhecido por Cote
d'Ivoire ebolavirus); e vírus Bundibugyo (Bundibugyo ebolavirus).
A quinta espécie, o vírus Reston (Reston ebolavirus), provocou
doenças em primatas não-humanos, mas não em humanos.
Os vírus Ébola estão presentes em vários países africanos. O Ébola
foi descoberto em 1976 perto do rio Ébola no território que
pertence actualmente à República Democrática do Congo. Desde
então, ocorrem surtos esporádicos em África. Ainda se desconhece o reservatório hospedeiro natural dos vírus Ébola. Contudo, com base em provas e na natureza de vírus semelhantes,
os investigadores acreditam que o vírus é veiculado por animais e que os morcegos são os hospedeiros mais prováveis. Quatro das cinco
subespécies surgem em animais hospedeiros originários de África. \\
A partir desse panorama, julgamos pertinente modelar os casos de Ébola causados pela infecção do vírus Ébola. O objetivo do modelo proposto é descrever a dinâmica de incidência da população da África que apresentam Ébola. 


\newpage
\pagestyle{normal}

\section*{Revisão da literatura}
Segundo a \emph{mathematical model of Ebola virus disease in Africa} \cite{Aqsa Nazir} a principal fonte para iniciar a doença foi "animal ”, em tal
maneira que quando um homem caçava por comida, seu contato acontecia com animais infectados (como macacos, chimpanzés e
morcegos frugívoros, etc.). A observação mencionada acima permitiu
para afirmarmos que o contato indireto pode ser uma das razões
para a propagação da doença . Má higiene e sanitária
condições são também uma das razões para a propagação do
Vírus Ebola na África.
Um modelo do tipo SIR (suscetível-infectado-recuperado) para o estudo da propagação do Ebola Virus Disease (EVD) é desenvolvido usando derivados conformáveis. Todas as maneiras possíveis de transmissão da doença é incorporada (direta ou indireta), como práticas funerárias, consumo de carne de caça contaminada e a contaminação ambiental, etc.  Descobrimos que a única situação livre de doença é a ausência de fluxo da doença do vírus Ebola do
ambiente. Também observamos que, ao adotar algumas estratégias, como o isolamento de infectados
indivíduos e enterro cuidadoso de cadáveres, a propagação de EVD pode ser controlada. \\
Estudando o Artigo, percebemos que a persistência e recorrência de EVD na África é devido a : \\
\begin{enumerate}
    \item Consumo de carne de caça contaminada.
    \item As cerimônias fúnebres.
    \item Poluente ambiental.
    \item Vômito, leite materno e urina, etc (Transmissão Direta).
    \item Objetos como roupas contaminadas etc (transmissão indireta).
\end{enumerate} 
\\
 Segundo o artigo de \emph{Mathematical model of Ebola transmission dynamics with relapse and reinfection} \cite{F.B. Agusto} a epidemia de Ebola ceifou a vida de mais de 11.300
pessoas e infectou mais de 28.500. A doença causou devastação sobre as famílias, comunidades e ao sistema econômico de saúde dos 3 países mais afetados (Guiné, Libéria e Serra
Leone).\\
A recuperação de EVD requer tanto humoral quanto imunidade mediada por células, e há variabilidade na reação imunológica dos indivíduos. Além disso, a variabilidade
na imunidade do hospedeiro pode determinar a suscetibilidade do hospedeiro a reinfecção. \\
Entendemos no artigo que a letalidade do caso de EVD variou de 25\% a 90\% no entando a média de letalidade foi de 50\%, altamente perigoso. 
\section*{Metodologia}
Os modelos matemáticos epidêmicos são de fundamental importância para a análise e compreensão sobre a dinâmica do processo de contágio, nesse caso, de doenças infecciosas, na sociedade. A fim de compreendermos a evolução dos casos de Ébola um modelo matemático compartimental foi usado para estimar o número de casos secundários gerados por um caso índice, na ausência ou presença de medidas de controle.

\section*{Formulação do modelo}
Iremos modelar a doença do vírus Ebola por um modelo compartimental. \\
Dito isso, iremos introduzir primeiro os nossos parâmetros com suas descrições do modelo que iremos apresentar. \\
\begin{center}
\begin{tabular}{||c c||} 
 \hline
 Variável & Descrição\\ [0.5ex] 
 \hline\hline
 $(t)$ & População de indivíduos suscetíveis  \\ 
 \hline
 $E(t)$ & População de indivíduos expostos  \\
 \hline
 $I_E$ & População de indivíduos sintomáticos no estágio inicial de infecção por EBOV \\
 \hline
 $I_L$ & População de indivíduos sintomáticos no estágio final da infecção por EBOV \\
 \hline
 $R_1(t),R_2(t)$ & População de indivíduos recuperados e imunes \\
 \hline
 $D(t)$ & População de indivíduos falecidos por Ebola \\[1ex]
 \hline
\end{tabular}
\end{center}
\begin{center}
\begin{tabular}{||c c||} 
 \hline
 Parâmetros & Descrição\\ [0.5ex] 
 \hline\hline
 $\beta$ & Taxa de contato (transmissão) efetiva  \\ 
 \hline
 $\Pi$ & Taxa de recrutamento  \\
 \hline
 $\mu$ & Taxa de mortalidade natural \\
 \hline
 $\tau$ & Parâmetros de modificação para infecciosidade \\
 \hline
 $\rho$ & Taxa de reativação de infecção\\
 \hline
 $\sigma$ & Taxa de progressão de indivíduos sintomáticos \\[1ex]
 \hline
 $\alpha$ & Taxa de progressão de indivíduos com sintomas iniciais\\
 \hline
 $\xi$ & Taxa de progressão de indivíduos recuperados para a classe imune\\
 \hline
 $h$ & Fração de indivíduos sintomáticos que se recuperaram\\
 \hline
 $\epsilon$ & Parâmetros de modificação de reinfecção\\
 \hline
 $\gamma$ & Taxa de recuperação de indivíduos sintomáticos\\
 \hline
 $\delta$ & Taxa de cremação / sepultamento de indivíduos falecidos com Ebola\\ [1ex]
 \hline
\end{tabular}
\end{center}
\textbf{Iniciando o modelo}. \vspace{3mm}\\
A nossa população total , $N_H(t)$ no tempo t é dividida em subpopulações mutuamente exclusivas de indivíduos que são suscetíveis($S(t)$), expostos ($E(t)$), indivíduos sintomático no estágio inicial de infecção por EVD ($I_E(t)$), indivíduos sintomático no estágio final da infecção por EVD ($I_L(t)$), se recuperaram e indivíduos imunes ($R_1(t), R_2(t)$) e indivíduos falecidos infectados com Ebola ($D(t)$). De modo a
\\ $$N(t) = S(t) + E(t) + I_E(t) + I_L(t) + R_1(t) + R_2(t) + D(t)$$
$$S(t) = \Pi -\lambda(I_E,I_L,R_1,D)S(t) - \mu S(t)$$
$$E(t) = \lambda(I_E,I_L,D)S(t) + \epsilon \lambda(I_E,I_L,R_1,D)R_1(t)- (\sigma + \mu)E(t),$$
$$I_E(t) = \sigma E(t)-(\alpha + \mu)I_E(t) + \rho R(t),$$
$$I_L(t) = \alpha I_E(t) - (\gamma + \mu)I_L(t),$$
$$R_1(t) = h \gamma I_L(t) - (\rho + \xi + \mu)R_1(t) - \epsilon \lambda(I_E,I_L,R_1,D)R_1(t),$$
$$R_2(t) = \xi R_1(t) - \mu R_2(t) ,$$
$$D(t) = (1 - h)\gamma I_L(t) - \delta D(t) ,$$
Onde , $\lambda(I_E,I_L,R_1,D) = \dfrac{\Beta (I_E + I_L + \tau_1 R_1 + \tau_2 D)}{S + E + I_E + I_L + R_1 + R_2 + D}$ é a taxa de infecção da doença. \vspace{3mm}\\
Pode-se vê que incluimos duas classes recuperadas( $R_1$ e $R_2$) e assumimos que os indivíduos na classe $R_1(t)$ podem sofrer reinfecção. Além disso, presumimos que esses indivíduos sejam capazes de transmitir o vírus, uma vez que os vírus podem persistir após a recuperação em partes do corpo e podem se espalhar através do sexo ou outros contatos com o sêmen. Já os da classe $R_2(t)$ experimentam imunidade vitalícia.
 
\newpage
\begin{thebibliography}{9}
\bibitem{F.B. Agusto}
F.B. Agusto (2016) \emph{Mathematical model of Ebola transmission dynamics with relapse and reinfection}.

\bibitem{Aqsa Nazir}
Aqsa Nazir , Naveed Ahmed , Umar Khan, Syed Tauseef Mohyud-Din , Kottakkaran Sooppy Nisar (2020) \emph{mathematical model of Ebola virus disease in Africa} 

\bibitem{Revista Ciencia & Inovação }
Revista Ciencia & Inovação - FAM - V.5, N.1 - JUN - (2020) \emph{principais métodos de diagnóstico
e tratamento da doença causada pelo vírus Ebola} 

\bibitem{Camila Soares de Souza}
Camila Soares de Souza (2017) \emph{previsão de surto epidêmico de Ebola abordagens probabilísticas e determinísticas}

\end{thebibliography}

\end{document}
