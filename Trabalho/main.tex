\documentclass[12pt,a4paper]{article}
\usepackage[utf8]{inputenc}
\usepackage[brazil]{babel}
\usepackage{amsmath}
\usepackage{listings}
\usepackage[utf8]{inputenc}
\usepackage{makeidx}
\usepackage{listings}
\usepackage{xcolor}
\usepackage{graphicx,url}
\usepackage{subfigure}
\usepackage{float}
\definecolor{codegreen}{rgb}{0,0.6,0}
\definecolor{codegray}{rgb}{0.5,0.5,0.5}
\definecolor{codepurple}{rgb}{0.58,0,0.82}
\definecolor{backcolour}{rgb}{0.95,0.95,0.92}

\lstdefinestyle{mystyle}{
    backgroundcolor=\color{backcolour},   
    commentstyle=\color{codegreen},
    keywordstyle=\color{magenta},
    numberstyle=\tiny\color{codegray},
    stringstyle=\color{codepurple},
    basicstyle=\ttfamily\footnotesize,
    breakatwhitespace=false,         
    breaklines=true,                 
    captionpos=b,                    
    keepspaces=true,                 
    numbers=left,                    
    numbersep=5pt,                  
    showspaces=false,                
    showstringspaces=false,
    showtabs=false,                  
    tabsize=2
}

\lstset{style=mystyle,
literate      =        % Support additional characters
      {á}{{\'a}}1  {é}{{\'e}}1  {í}{{\'i}}1 {ó}{{\'o}}1  {ú}{{\'u}}1
      {Á}{{\'A}}1  {É}{{\'E}}1  {Í}{{\'I}}1 {Ó}{{\'O}}1  {Ú}{{\'U}}1
      {à}{{\`a}}1  {è}{{\`e}}1  {ì}{{\`i}}1 {ò}{{\`o}}1  {ù}{{\`u}}1
      {À}{{\`A}}1  {È}{{\'E}}1  {Ì}{{\`I}}1 {Ò}{{\`O}}1  {Ù}{{\`U}}1
      {ä}{{\"a}}1  {ë}{{\"e}}1  {ï}{{\"i}}1 {ö}{{\"o}}1  {ü}{{\"u}}1
      {Ä}{{\"A}}1  {Ë}{{\"E}}1  {Ï}{{\"I}}1 {Ö}{{\"O}}1  {Ü}{{\"U}}1
      {â}{{\^a}}1  {ê}{{\^e}}1  {î}{{\^i}}1 {ô}{{\^o}}1  {û}{{\^u}}1
      {Â}{{\^A}}1  {Ê}{{\^E}}1  {Î}{{\^I}}1 {Ô}{{\^O}}1  {Û}{{\^U}}1
      {œ}{{\oe}}1  {Œ}{{\OE}}1  {æ}{{\ae}}1 {Æ}{{\AE}}1  {ß}{{\ss}}1
      {ç}{{\c c}}1 {Ç}{{\c C}}1 {ø}{{\o}}1  {å}{{\r a}}1 {Å}{{\r A}}1
      {ã}{{\~a}}1  {õ}{{\~o}}1  {Ã}{{\~A}}1 {Õ}{{\~O}}1
      {ñ}{{\~n}}1  {Ñ}{{\~N}}1  {¿}{{?`}}1  {¡}{{!`}}1
}
\input{settings/Configuracoes_do_Preambulo}

\begin{document}
 

%\maketitle
\thispagestyle{primeira}



\section*{Resumo}

\section*{Introdução}
O Ebola Virus Disease (EVD) , anteriormente conhecido como febre hemorrágica Ébola, é uma doença grave e fatal causada pela infecção com uma das espécies do vírus do Ébola. O Ébola pode provocar doenças em humanos e primatas não-humanos (macacos, gorilas e chimpanzés).
O Ébola é provocado por um vírus da família Filoviridae, género
Ebolavirus. Já foram descobertas cinco espécies do vírus do Ébola, quatro delas provocam doenças nos humanos: Vírus do Ébola (Zaire
ebolavirus); vírus do Sudão (Sudan ebolavirus); vírus da floresta
Tai (Tai Forest ebolavirus, anteriormente conhecido por Cote
d'Ivoire ebolavirus); e vírus Bundibugyo (Bundibugyo ebolavirus).
A quinta espécie, o vírus Reston (Reston ebolavirus), provocou
doenças em primatas não-humanos, mas não em humanos.\\
Os vírus Ébola estão presentes em vários países africanos,
foi descoberto em 1976 perto do rio Ébola no território que
pertence actualmente à República Democrática do Congo. Desde
então, ocorrem surtos esporádicos na África. Ainda se desconhece o reservatório hospedeiro natural dos vírus Ébola, contudo, com base em provas e na natureza de vírus semelhantes,
os investigadores acreditam que o vírus é veiculado por animais e que os morcegos são os hospedeiros mais prováveis, quatro das cinco subespécies surgem em animais hospedeiros originários da África. \\
A partir desse panorama, julgamos pertinente modelar os casos de Ébola causados pela infecção do vírus Ébola. O objetivo do modelo proposto é descrever a dinâmica de incidência da população da África que apresentam Ébola. 


\newpage
\pagestyle{normal}

\section*{Revisão da literatura}
Segundo a \emph{mathematical model of Ebola virus disease in Africa} \cite{Aqsa Nazir} a principal fonte para iniciar a doença foi "animal ”, em tal
maneira que quando um homem caçava por comida, seu contato acontecia com animais infectados (como macacos, chimpanzés e
morcegos frugívoros, etc.). A observação mencionada acima permitiu
para afirmarmos que o contato indireto pode ser uma das razões
para a propagação da doença . Má higiene e condições sanitária são também uma das razões para a propagação do Vírus Ebola na África.\\
Um modelo do tipo SIR (suscetível-infectado-recuperado) para o estudo da propagação do Ebola Virus Disease (EVD) é desenvolvido usando derivadas conformáveis. Todas as maneiras possíveis de transmissão da doença é incorporada (direta ou indireta), como práticas funerárias, consumo de carne de caça contaminada e a contaminação ambiental, etc.  Descobrimos que a única situação livre de doença é a ausência de fluxo da doença do vírus Ebola do
ambiente. Também observamos que, ao adotar algumas estratégias, como o isolamento de indivíduos infectados
 e enterro cuidadoso de cadáveres, a propagação de EVD pode ser controlada. \\
Foi adicionado um termo importante que é a possibilidade de \textbf{nascimento de um indivíduo infectado e a migração de um indivíduo infectado para a população existente}. 
Estudando o Artigo, percebemos que a persistência e recorrência de EVD na África é devido a : \\
\begin{enumerate}
    \item Consumo de carne de caça contaminada.
    \item As cerimônias fúnebres.
    \item Poluente ambiental.
    \item Vômito, leite materno e urina, etc (Transmissão Direta).
    \item Objetos como roupas contaminadas etc (transmissão indireta).
\end{enumerate} 
\\
 Segundo o artigo de \emph{Mathematical model of Ebola transmission dynamics with relapse and reinfection} \cite{F.B. Agusto} 
A recuperação de EVD requer tanto humoral quanto imunidade  por células, e há variabilidade na reação imunológica dos indivíduos. Além disso, a variabilidade
na imunidade do hospedeiro pode determinar a suscetibilidade do hospedeiro a reinfecção. \\
Entendemos no artigo que a letalidade do caso de EVD variou de 25\% a 90\% no entando a média de letalidade foi de 50\%, altamente perigoso. \\
O modelo Ebola com recidiva da doença e reinfecção é localmente assintoticamente estável quando o número é menor que a unidade. 
O modelo exibe, na presença de reinfecção da doença , o fenômeno de bifurcação para trás, onde o equilíbrio livre da doença estável coexiste com um equilíbrio endêmico estável, quando o número de reprodução é menor que a unidade

\section*{Metodologia}
Os modelos matemáticos epidêmicos são de fundamental importância para a análise e compreensão sobre a dinâmica do processo de contágio, nesse caso, de doenças infecciosas, na sociedade. A fim de compreendermos a evolução dos casos de Ébola um modelo  SIRDP conformável onde P está denotando o compartimento do meio ambiente, e pode haver a possibilidade de um feto poder pegar a infecção de sua mãe no útero e pode ser adicionado diretamente à população infectada. Similharmente, há uma chance de migração de pessoa infectada para a população em particular. Portanto, também é uma fonte de adição à população de indivíduos infectados

\section*{Formulação do modelo}
Para investigar a propagação do Ebola, temos as seguintes suposições:\\
Uma das causas da propagação da infecção é sobre os indivíduos humanos falecidos, que podem transmitir durante as cerimônia de enterro\\
Outra causa pode ser pelo meio ambiente por meio de urina e fezes de indivíduos infectados\\
E ainda pelo ambiente devido o consumo de carne de caça contaminada. O surto de EDV duraram cerca de dois anos, o que ocorreu uma nova adição na população dos infectados por meio dos nascimentos ou migração de indivíduos infectados.\vspace{2mm}\\
Então foi proposto um modelo matemático com base o que foi dito acima da seguinte forma:\\

\begin{center}
\begin{tabular}{||c c||} 
 \hline
 Variável & Descrição\\ [0.5ex] 
 \hline\hline
 $F$ & População de indivíduos suscetíveis  \\ 
 \hline
 $L$ & População de indivíduos infectados  \\
 \hline
 $Q$ & População de indivíduos  recuperados\\
 \hline
 $W$ & População de indivíduos falecidos infectados \\
 \hline
 $G$ & Patógenos do vírus no ambiente \\
 \hline
\end{tabular}
\end{center}
\begin{center}
\begin{tabular}{||c c||} 
 \hline
 Parâmetros & Descrição\\ [0.5ex] 
 \hline\hline
 $a_1$ & A taxa de recrutamento de suscetíveis  \\ 
 \hline
 $a_2$ & A taxa de recrutamento de infectados  \\
 \hline
 $b$ & Taxa de enterro de falecido \\
 \hline
 $\beta_1$ & Taxa de contato (efetivo) de humano infeccioso \\
 \hline
 $\beta_2$ & Taxa de contato (efetivo) do falecido\\
 \hline
 $\lambda$ & Taxa de contato (efetivo) do Ebolavírus \\[1ex]
 \hline
 $\sigma$ & Taxa de mortes naturais de humanos\\
 \hline
 $\eta$ & Taxa de mortes de indivíduos humanos devido a infecção\\
 \hline
 $\mu$ & Taxa de recrutamento de EVD no ambiente\\
 \hline
 $\xi$ & Taxa de queda de humanos infectados\\
 \hline
 $\alpha$ & Taxa de queda de humanos falecidos\\ [1ex]
 \hline
 $\gamma$ & Taxa de recuperados da doença\\
 \hline
\end{tabular}
\end{center}

\vspace{4mm}

\includegraphics[scale=0.6]{images/fluxograma.png}
\begin{align}
    
    &\dfrac{dF(t)}{dt} = a_1 - (\beta_1 L(t) + \beta_2 W(t) + \lambda G(t)) F(t) - \sigma F(t) \vspace{2mm} \\
    
    &\dfrac{dL(t)}{dt} = a_2 + (\beta_1 L(t) + \beta_2 W(t) + \lambda G(t)) F(t) - (\sigma + \eta + \gamma) L(t)\vspace{2mm}\\
    
    &\dfrac{dQ(t)}{dt} = \gamma L(t) - \sigma Q(t)\vspace{2mm}\\
    
    &\dfrac{dW(t)}{dt} = (\sigma + \eta) L(t) - bW(t)\vspace{2mm}\\
    
    &\dfrac{dG(t)}{dt} = \mu + \xi L(t) + \alpha W(t) - \delta G(t)
\end{align}

\vspace{5mm}

\noindent \textbf{Definição:}\\
    Seja \textit{g} uma função com domínio real, então a derivada fracionária conformável de g , com ordem $\theta$, é definida como \vspace{2mm}:\\
    $B_\theta (g)(t) = \dfrac{g(t + \epsilon t^{1-\theta}) - g(t)}{\epsilon}$, $\forall t > 0$ e $\theta \in (0,1]$\\.
    
    \noindent Ou ainda, se g é derivável, então:\\
    $B_\theta (g)(t) = t^{1-\theta} \dfrac{dg}{dt}$\\
    
    \noindent Explicação melhor pode ser encontrada em \cite{R. Khalil} em que R. Khalil e outros autores abordam as sefinições de derivadas fracionárias.\\
    
    
    
    \vspace{3mm}
    \noindent Então vamos definir todas as equações feitas usando derivada fracionária conformável da seguinte forma:
    
    \begin{align}
    
    & B_\theta (F)(t) = a_1 - (\beta_1 L(t) + \beta_2 W(t) + \lambda G(t)) F(t) - \sigma F(t) \vspace{2mm} \\
    
    & B_\theta (L)(t) = a_2 + (\beta_1 L(t) + \beta_2 W(t) + \lambda G(t)) F(t) - (\sigma + \eta + \gamma) L(t)\vspace{2mm}\\
    
    & B_\theta (Q)(t) = \gamma L(t) - \sigma Q(t)\vspace{2mm}\\
    
    & B_\theta (W)(t) = (\sigma + \eta) L(t) - bW(t)\vspace{2mm}\\
    
    & B_\theta (G)(t) = \mu + \xi L(t) + \alpha W(t) - \delta G(t)
\end{align}
\noindent Em que $B_\theta$ é um operador simbolizando a derivada conformável da função com a ordem da derivada. Agora usando a outra definição, teremos:

    \begin{align}
    
    & t^{1-\theta} (F)'(t) = a_1 - (\beta_1 L(t) + \beta_2 W(t) + \lambda G(t)) F(t) - \sigma F(t) \vspace{2mm} \\
    
    & t^{1-\theta} (L)'(t) = a_2 + (\beta_1 L(t) + \beta_2 W(t) + \lambda G(t)) F(t) - (\sigma + \eta + \gamma) L(t)\vspace{2mm}\\
    
    & t^{1-\theta} (Q)'(t) = \gamma L(t) - \sigma Q(t)\vspace{2mm}\\
    
    & t^{1-\theta} (W)'(t) = (\sigma + \eta) L(t) - bW(t)\vspace{2mm}\\
    
    & t^{1-\theta} (G)'(t) = \mu + \xi L(t) + \alpha W(t) - \delta G(t)
\end{align}

\noindent Colocando $t^{1-\theta}$ para o outro lado , teremos:

\begin{align}
    
    & (F)'(t) = t^{\theta-1}(a_1 - (\beta_1 L(t) + \beta_2 W(t) + \lambda G(t)) F(t) - \sigma F(t)) \vspace{2mm} \\
    
    & (L)'(t) = t^{\theta-1} (a_2 + (\beta_1 L(t) + \beta_2 W(t) + \lambda G(t)) F(t) - (\sigma + \eta + \gamma) L(t))\vspace{2mm}\\
    
    & (Q)'(t) = t^{\theta-1} (\gamma L(t) - \sigma Q(t))\vspace{2mm}\\
    
    & (W)'(t) = t^{\theta-1} (\sigma + \eta) L(t) - bW(t))\vspace{2mm}\\
    
    & (G)'(t) = t^{\theta-1} (\mu + \xi L(t) + \alpha W(t) - \delta G(t))
\end{align}

\noindent Com as condições :\\
\begin{align*}
    F(0) = F_0 & & & & & & & & & & & & & & & & & & & & & &\\
    L(0) = L_0 \\
    Q(0) = Q_0 \\
    W(0) = W_0 \\
    G(0) = G_0
\end{align*}

\vspace{4mm}
Fazendo $Z = F + L + Q$ que é a soma das populações vivas , teremos que:\vspace{2mm}\\
$\dfrac{d Z(t)}{dt} = t^{\theta - 1}(a_1 + a_2 - \eta L - \sigma Z)$

\section*{Resultados}

Suponha que as condições iniciais satisfazem o seguinte:\\
$Z_0 \leq Z_m$, $W_0 \leq W_m$ e $G_0 \leq G_m$ , em que \vspace{2mm}\\
$Z_m = \dfrac{a_1 + a_2}{\sigma}$ , $W_m = \dfrac{(\sigma + \eta)(a_1 + a_2)}{b\sigma}$ e $G_m = \dfrac{\sigma b \mu + b\xi(a_1 + a_2) + \alpha(\sigma + \eta)(a_1 + a_2)} {b \delta \sigma}$

\vspace{5mm}
\noindent Assim iremos mostrar que vale $Z(t) \leq Z_m$ , $W(t) \leq W_m$ e $G(t) \leq G_m$ $\forall t$ .\\

\vspace{4mm}
\noindent Como $\dfrac{dZ(t)}{dt} = t^{\theta - 1}(a_1 + a_2 - \sigma Z - \eta L)$ e como $L(t) \geq 0$ , então \vspace{1mm}\\ $\dfrac{dZ(t)}{dt} \leq t^{\theta - 1}(a_1 + a_2 - \sigma Z) $ , agora temos a seguinte \textit{desigualdade de Gronwall}:\vspace{3mm}\\
Seja $u(t)$ uma função não negativa e diferenciável, que satisfaz:\\
$u'(t) \leq f(t) u(t) + g(t)$, onde $f(t)$ e $g(t)$ são funções integráveis não negativas, então :\vspace{2mm}\\
$u(t) \leq e^{\int_0 ^t f(x) dx} [u(0) + \int_0 ^t g(y)dy]$.\\

\vspace{3mm}
\noindent Assim podemos ajeitar a derivada do $Z(t)$ da seguinte forma:\\
$\dfrac{d Z(t)}{dt} \leq -t^{\theta - 1} \sigma Z(t) + (a_1 + a_2)t^{\theta - 1}$, em que $f(x) = -x^{\theta - 1} \sigma$ e $g(y) = (a_1 + a_2)y^{\theta - 1}$
\newpage
\noindent Assim teremos:\\

$e^{\int_0^t f(x) dx} = e^{-\int_0^t (x^{\theta - 1}\sigma) dx} = e^{-\frac{\sigma}{\theta} t^\theta}$ e também temos:\\

$\int_0^t g(y) dy = \int_0^t (a_1+a_2)y^{\theta-1}dy = \frac{(a_1 + a_2)}{\theta}t^\theta $. Portanto teremos :\\

$Z(t) \leq e^{-\frac{\sigma}{\theta} t^\theta} [Z(0) + \frac{(a_1 + a_2)}{\theta}t^\theta]$, rearrumando teremos:\vspace{2mm}\\
$Z(t) \leq \frac{a_1 + a_2}{\sigma} + (Z(0) - \frac{a_1 + a_2}{\sigma})e^{-\frac{t^\theta \sigma}{\theta}} \Longrightarrow Z(t) \leq Z_m + (Z(0) - Z_m) e^{\frac{-t^\theta \sigma}{\theta}}$\\

Assim , como $Z(0) \leq Z_m$, então teremos \boxed{$Z(t) \leq Z_m$}. \vspace{1mm}\\
De modo análogo teremos $L(t) \leq Z_m$, $W(t) \leq W_m$ e $G(t) \leq G_m$

\vspace{5mm}

\noindent Portanto a gente chegou em um sistema dinâmico no seguinte conjunto compacto:\\

$\Omega = {(F(t),L(t),Q(t),W(t),G(t)):$\\$ Z(t) \leq \dfrac{a_1 + a_2}{\sigma} \\W(t) \leq \dfrac{(\sigma + \eta)(a_1 + a_2)}{b \sigma}\\ G(t) \leq \dfrac{\sigma b \mu + b\xi (a_1+a_2) + \alpha ( \sigma + \eta)(a_1 + a_2)}{b \delta \sigma}}$

\vspace{5mm}
\subsection*{Equilíbrio}
\noindent Agora vamos encontrar o equilíbrio do modelo , suponha que $(F_1,L_1,Q_1,W_1,G_1)$ sejam os pontos de equilíbrio do modelo, então teremos:
\begin{align*}
    &a_1 - (\beta_1 L_1 + \beta_2 W_1 + \gamma G_1) F_1 - \sigma F_1 = 0 & & & & & & & & & & & & & &\\
     &a_2 + (\beta_1 L_1 + \beta_2 W_2 + \lambda G_1) F_1 - (\sigma + \eta + \gamma) L_1 = 0\\
     &\gamma L_1 - \sigma Q_1  = 0\\
     &(\sigma + \eta) L_1 - bW_1 = 0\\
     & \mu + \xi L_1 + \alpha W_1 - \delta G_1 = 0
\end{align*}

\newpage

\noindent Resolvendo o sistema , encontraremos:\vspace{2mm}\\
$F_1 = \dfrac{(a_1+a_2) - (\sigma + \eta + \gamma)L_1}{\sigma}$\\
$Q_1 = \dfrac{\gamma L_1}{\sigma}$\\
$W_1 = \dfrac{(\sigma + \eta)L_1}{b}$\\
$G_1 = \dfrac{b \mu + (b \xi + \alpha \eta + \alpha \sigma)L_1}{b \delta}$

\subsection*{Calculo do R_0}
O Número Básico de Reprodução, geralmente representado por $R_0$ , é o número esperado de novos
casos de infecção gerados a partir da introdução de um individuo infectado em uma população
suscetível. É de suma importância em um modelo epidêmico pois fornece algumas informações
importantes, entre as quais destaca-se o fato de que se ele for inferior a 1 a doença tende a sumir
com o tempo (um indivíduo infectado produz menos de um novo indivíduo infectado ao longo de seu período infeccioso), e se for superior a 1 a tendência é que a doença se espalhe na população ( cada indivíduo infectado produz, em média, mais de uma nova infecção, e a doença pode invadir a população). Em \cite{R0}, define-se $R_0$  como o raio espectral da matriz da próxima geração
. Logo, \\ \vspace{6} \\
 $X = \begin{bmatrix}
- \frac{\beta_1 a_1}{\sigma} & 0 & - \frac{\beta_2 a_1}{\sigma} & - \frac{\lambda a_1}{\sigma} \\
0 & 0 & 0 & 0 \\
0 & 0 & 0 & 0 \\
0 & 0 & 0 & 0 \\
\end{bmatrix}$ e $Y = \begin{bmatrix}
\sigma + \eta + \gamma & 0 & 0 & 0\\
- \gamma & \sigma & 0 & 0\\
- \sigma - \eta & 0 & b & 0\\
- \xi  & 0 & - \alpha & \delta
\end{bmatrix}$ \\ \vspace{3} \\Onde $X$ e $Y$ são respectivamente matrizes de transmissão e transição , assim, fazendo o produto delas chegamos que\vspace{2mm}\\
$X Y^{-1} = \begin{bmatrix}
\frac{\beta_1 a_1}{(\sigma + \eta + \gamma)\sigma} + \frac{\beta_2 a_1(\sigma + \eta)}{(\sigma + \eta + \gamma)\sigma b} + \frac{\lambda a_1(\alpha \eta + \alpha \sigma + b\xi)}{(\sigma + \eta + \gamma)\sigma b \delta} & 0 & \frac{\beta_2 a_1}{ab} +  \frac{\lambda a_1 \alpha}{ab\delta} & \frac{\lambda a_1}{a \gamma}  \\
0 & 0 & 0 & 0\\
0 & 0 & 0 & 0\\
0 & 0 & 0 & 0\\
\end{bmatrix}$ \vspace{3mm}\\ E com isso, chegamos que o número básico de reprodução será :\vspace{2mm}\\ $R_0 = \dfrac{\beta_1 a_1}{(\sigma + \eta + \gamma)\sigma} + \dfrac{\beta_2 a_1(\sigma + \eta)}{(\sigma + \eta + \gamma)\sigma b} + \dfrac{\lambda a_1(\alpha \eta + \alpha \sigma + b\xi)}{(\sigma + \eta + \gamma)\sigma b \delta}$


\subsection{Resultado e Discussão da solução do modelo}

\noindent O sistema de equações diferenciais, foi resolvido numericamente para obter soluções aproximadas



\begin{thebibliography}{9}
\bibitem{F.B. Agusto}
F.B. Agusto (2016) \emph{Mathematical model of Ebola transmission dynamics with relapse and reinfection}.

\bibitem{Aqsa Nazir}
Aqsa Nazir , Naveed Ahmed , Umar Khan, Syed Tauseef Mohyud-Din , Kottakkaran Sooppy Nisar (2020) \emph{mathematical model of Ebola virus disease in Africa} 

\bibitem{Revista Ciencia & Inovação }
Revista Ciencia & Inovação - FAM - V.5, N.1 - JUN - (2020) \emph{principais métodos de diagnóstico
e tratamento da doença causada pelo vírus Ebola} 

\bibitem{Camila Soares de Souza}
Camila Soares de Souza (2017) \emph{previsão de surto epidêmico de Ebola abordagens probabilísticas e determinísticas}

\bibitem{R. Khalil}
R. Khalil, M. Al Horani, A. Yousef, M. Sababheh. A new definition of fractional derivative, J. Comput. Appl. Math. 264 (2014) 65–7

\bibitem{R0}
P. Van Den Driessche, J. Watmough, Reproduction numbers
and sub-threshold endemic equilibria for compartmental models
of disease transmission, Math. Biosci. Elsevier, 180(1–2), 29–48,
01-Nov-2002.
\end{thebibliography}

\end{document}
